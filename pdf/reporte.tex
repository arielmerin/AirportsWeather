\documentclass[letterpaper]{article}
\usepackage[utf8]{inputenc}
\usepackage[spanish, mexico]{babel}
\usepackage{amssymb, amsmath}
\usepackage{graphicx}
\usepackage[margin=1.5cm,
vmargin={1.5cm,0.7cm},
includefoot]{geometry}
\usepackage{amsthm}
\usepackage{dsfont}
\usepackage{mathtools}
\usepackage{graphicx}
\usepackage{algorithmic}
\usepackage[linesnumbered,ruled,vlined]{algorithm2e}
\begin{document}

\setlength{\unitlength}{1cm}
\thispagestyle{empty}
\begin{picture}(19,3)
\put(-0.5,1.2){\includegraphics[scale=.20]{img/unam1.png}}
\put(16,1){\includegraphics[scale=.29]{img/fciencias1.png}}
\end{picture}

\begin{center}
	\vspace{-114pt}
	\textbf{\large Proyecto 01}\\
	\textbf{ Semestre 2021-1}\\
	Prof. José Galaviz Casas\\
	Ayud. María Ximena Lezama \\
	\textbf{Modelado y programación}\\[0.15cm]
	Kevin Ariel Merino Peña\footnote{Número de cuenta 317031326} Armando Abraham Aquino Chapa\footnote{Número de cuenta n}\\
	\today
\end{center}
\vspace{-10pt}
\rule{19cm}{0.3mm}

\section{Definición del problema}
\section{Análisis del problema y selección de la mejor alternativa. }
Para comenzar con el análisis del problema,uno de los aspectos fundamentales fue hacernos preguntas del tipo ¿qué es lo que queremos obtener?, ¿cuáles son los datos que tenemos para obtenerlo?, ¿qué hace que el resultado obtenido resuelva el problema?. También, cabe destacar que al analizar el problema, uno de nuestros "métodos" fue descomponerlo en subproblemas más simples para así ir solucionando cada subproblema a la vez y obtener una solución óptima. Por lo tanto, se determinó que lo más adecuado para resolverlo era obtener las siguientes clases:
\begin{itemize}
	\item \textbf{Clase Weather:}.
	\begin{itemize}
		\item Esta clase sería la encargada de realizar las distintas peticiones al servidor para obtener todos los datos correspondientes al clima. Aquí también modelaríamos nuestro caché.
		\item Manejar los posibles errores, como el exceso de numero peticiones en un lapso de tiempo, para no obtener ningún problema con el servidor,y manejar los errores donde no es posible consultar la información
		\item Otorgar de un formato correcto y legible a la salida del programa
	\end{itemize}
	\item \textbf{Clase CSVReader}
	\begin{itemize}
		\item Su función principal sería leer los archivos \textit{csv} otorgados por el aeropuerto de la Ciudad de México.  
		\item Preprocesar cierto tipo de información que si es esencial de la que es redundante o no tiene utilidad para nuestros objetivos.
		\item Ordenar información (de los archivos. \textit{csv} que se nos fueron proveídos) para que de esta forma no obtengamos elementos repetidos, y así evitar realizar peticiones de más.
	\end{itemize}
	\item \textbf{Clase main.}
	\begin{itemize}
		\item Cómo su nombre lo indica, en esta clase se concretarían las peticiones y formatos de respuestas a nuestros clientes.
		\item Procesar las peticiones no repetidas para posteriormente enviarlas al servidor.
		\item Asimismo, verificaríamos la consistencia del archivo enviado como parámetro para evitar cualquier tipo de inconveniente y que nuestro programa no colapse.
	\end{itemize}
\end{itemize}
Una vez visto el análisis del problema, respondamos la siguiente pregunta: ¿La forma en qué hemos modelado el problema es la mejor alternativa?.\\
Si, es una de las mejores alternativas, ya que pueden asegurar que el problema se resuelve de una forma: eficiente, es decir, sin consumir la mínima cantidad de recursos, pero tampoco abusando de ellos. El programa es amigable con el usuario, porque a pesar de no ser interactivo, si tiene un muy buen formato el cual permite que el contenido sea legible y la información esté ordenada correctamente. También, el programa es tolerable a fallos. Aunque no es totalmente robusto, si tomamos en cuenta los errores más comunes y naturales a la hora de manejar el programa. \\
Más adelante mencionaremos cuales son los planes a futuro que tenemos para este proyecto, ya sea para una mejora o para mantenimiento.\\

\newpage
\section{Pseudocódigo}

	\subsection*{CSVReader.py}
	
	\begin{algorithm}
		\SetKwData{Left}{left}\SetKwData{This}{this}
		\SetKwData{Up}{up}
		\SetKwFunction{Union}{Union}
		\SetKwFunction{FindCompress}{FindCompress}
		\SetKwInOut{Input}{Entrada}
		\SetKwInOut{Output}{Salida}
		\SetAlgorithmName{Función}{0}{list of algorithms name}
		\Input{Nombre de un archivo (ruta)}
		\Output{Lista de coordenadas no repetidas}
		\BlankLine
		\While{Archivo(nombre dado) esté abierto}{
		\ForEach{renglones $\leftarrow$ documento}{
				\If{(latitud , longitud ) \textbf{no} están en  la lista}{
				Agreagrlas}
		}
	}
	\caption{read\_no\_repeated\_coordinates}
	\end{algorithm}
	
	\begin{algorithm}
		\SetKwData{Left}{left}\SetKwData{This}{this}
		\SetKwData{Up}{up}
		\SetKwFunction{Union}{Union}
		\SetKwFunction{FindCompress}{FindCompress}
		\SetKwInOut{Input}{Entrada}
		\SetKwInOut{Output}{Salida}
		\SetKwProg{try}{try}{:}{}
		\SetKwProg{catch}{catch}{:}{end}
		\SetAlgorithmName{Función}{0}{list of algorithms name}
		\Input{Nombre de un archivo (ruta)}
		\Output{Lista de diccionarios con vuelos}
		\BlankLine
		\try{}{
				Abrir ruta\\
				\For{linea $ \gets $ archivo}{
				linea $ \gets $ lista}
		}
		\catch{FileNotFoundError}{
			\textbf{muesta}	 Error, escribe una ruta válida\\\textbf{exit}
		}
		\catch{FileExistsError}{
		\textbf{muesta}	 Error, archivo válido\\\textbf{exit}
		}
	\caption{read\_csv\_file}
	\end{algorithm}
	
	\begin{algorithm}
		\SetKwData{Left}{left}\SetKwData{This}{this}
		\SetKwData{Up}{up}
		\SetKwFunction{Union}{Union}
		\SetKwFunction{FindCompress}{FindCompress}
		\SetKwInOut{Input}{Entrada}
		\SetKwInOut{Output}{Salida}
		\SetKwProg{with}{with}{:}{}
		\SetKwProg{catch}{catch}{:}{end}
		\SetAlgorithmName{Función}{0}{list of algorithms name}
		\Input{Nombre de un archivo (ruta)}
		\Output{Lista de cabeceras}
		\BlankLine
		\with{}{
			lector $ \gets $ \textbf{Leer primera linea}( ruta)
		}
	\caption{read\_headers}
	\end{algorithm}
	\newpage
	\subsection*{main.py}
	
	\begin{algorithm}
		\SetKwData{Left}{left}\SetKwData{This}{this}
		\SetKwData{Up}{up}
		\SetKwFunction{longitud}{longitud}
		\SetKwFunction{vf}{validate\_file}
		\SetKwFunction{readHeaders}{read\_headers}
		\SetKwInOut{Input}{Entrada}
		\SetKwInOut{Output}{Salida}
		\SetKwProg{with}{with}{:}{}
		\SetKwProg{catch}{catch}{:}{end}
		\SetAlgorithmName{Función}{0}{list of algorithms name}
		\Input{Nombre de un archivo (ruta) pasados como argumento al programa}
		\BlankLine
		\If{longiutd del argumento no es 2}{
		\textbf{muestra:} Error \\Debe indicar la ruta a un archivo csv\\ \textbf{salir}}
		\If{no coincide la extensión .csv}{\textbf{muestra: } Error, sólo admito archivos csv \\\textbf{salir}}
		cabezera $ \gets $ nombres de listas admitidas\\
		entrada\_cabecera $ \gets $ \readHeaders{argumento[1]}{}\\
		\If{\longitud{entrada\_cabecera} no es igual a \longitud{cabezera}}{
		\textbf{muestra:} ERROR El archivo csv debe tener los siguientes encabezados: origin, destination, origin\_latitude, origin\_longitude, destination\_latitude, destination\_longitude \\
		\textbf{salir}}
		\ForEach{cabeza in entrada\_cabezera}{\If{cabeza no está en cabezera}{\textbf{muestra:} ERROR El archivo csv debe tener los siguientes encabezados: origin, destination, origin\_latitude, origin\_longitude, destination\_latitude, destination\_longitude\\ \textbf{salir} }}
	\caption{validate\_file}
	\end{algorithm}
	
	\begin{algorithm}
		\SetKwData{Left}{left}\SetKwData{This}{this}
		\SetKwData{Up}{up}
		\SetKwFunction{makeapirequestbycoordinates}{make\_api\_request\_by\_coordinates}
		\SetKwFunction{format}{format}
		\SetKwFunction{setdefault}{setdefault}
		\SetKwFunction{vf}{validate\_file}
		\SetKwFunction{readcsvfile}{read\_csv\_file}
		\SetKwFunction{readnorepeatedcoordinates}{read\_no\_repeated\_coordinates}
		\SetKwInOut{Input}{Entrada}
		\SetKwInOut{Output}{Salida}
		\SetKwProg{with}{with}{:}{}
		\SetKwProg{catch}{catch}{:}{end}
		\SetAlgorithmName{Función}{0}{list of algorithms name}
		\Input{Nombre de un archivo (ruta)}
		\BlankLine
		%%%%%%%%%%%%%%%%%%%%%%%%%%%%%%
		\vf{argumentos al correr el programa}{}\;
		entradas $ \gets $ \readcsvfile{argumento al correr el programa}\\
		%%%%%%%5
		solicitudes\_no\_repetidas $ \gets $ \readnorepeatedcoordinates{argumentos al iniciar}\\
		\ForEach{solicitud \textbf{en} solicitudes\_no\_repetidas}{
		peticion $ \gets $ \makeapirequestbycoordinates{solicitud[0], solicitud[1]}\\
		peticiones $ \gets $ \setdefault{solicitud, peticion}\\
		}	
		\ForEach{entrada \textbf{en} entradas}{\textbf{muesta: } Datos del clima ;) con formato bonito }
	\caption{run}
	\end{algorithm}
	
	\newpage
	\subsection*{Weather.py}
	\begin{algorithm}
		\SetKwData{Left}{left}\SetKwData{This}{this}
		\SetKwData{Up}{up}
		\SetKwFunction{makeapirequestbycoordinates}{make\_api\_request\_by\_coordinates}
		\SetKwFunction{get}{get}
		\SetKwInOut{Input}{Entrada}
		\SetKwInOut{Output}{Salida}
		\SetKwProg{with}{with}{:}{}
		\SetKwProg{catch}{catch}{:}{end}
		\SetAlgorithmName{Función}{0}{list of algorithms name}
		\Input{latitud, lontigud}
		\Output{llamada a función parse\_weather\_info }
		\BlankLine
		%%%%%%%%%%%%%%%%%%%%%%%%%%%%%%
		\If{contador $ > $ 59}{contador $ \gets $ 0\\ esperar 1 minuto para continuar}
		\get{url + latitud y longitud dadas}{}\\
		contador $ \gets $ contador + 1
	\caption{make\_api\_request\_by \_coordinates}
	\end{algorithm}
	
	\begin{algorithm}
		\SetKwData{Left}{left}\SetKwData{This}{this}
		\SetKwData{Up}{up}
		\SetKwFunction{getLocalzone}{get\_localzone()}
		\SetKwFunction{fromTimeStamp}{fromtimestamp}
		\SetKwFunction{strftime}{strftime}
		\SetKwInOut{Input}{Entrada}
		\SetKwInOut{Output}{Salida}
		\SetAlgorithmName{Función}{0}{list of algorithms name}
		\Input{Número de fecha y hora (unix)}
		\Output{cadena de texto con hora en formato 12 hrs}
		\BlankLine
		%%%%%%%%%%%%%%%%%%%%%%%%%%%%%%
		\textbf{convierte\_fotante: } numero dado\\
		local\_timezone $ \gets $ \getLocalzone{ }\\
		local\_time $ \gets $ \fromTimeStamp{flotante, local\_timezone}\\
		\textbf{regresar: } local\_time con formato de 12 horas, (CODIGO DEL TIEMPO)
	\caption{formato\_de\_horas}
	\end{algorithm}
	
	\begin{algorithm}
		\SetKwData{Left}{left}\SetKwData{This}{this}
		\SetKwData{Up}{up}
		\SetKwFunction{makeapirequestbycoordinates}{make\_api\_request\_by\_coordinates}
		\SetKwFunction{get}{get}
		\SetKwInOut{Input}{Entrada}
		\SetKwInOut{Output}{Salida}
		\SetKwProg{try}{try}{:}{}
		\SetKwProg{catch}{catch}{:}{end}
		\SetAlgorithmName{Función}{0}{list of algorithms name}
		\Input{respuesta en formato json}
		\Output{llamada a función parse\_weather\_info }
		\BlankLine
		%%%%%%%%%%%%%%%%%%%%%%%%%%%%%%
		\try{ }{
		extraer información del archivo json con las llaves proporcionadas por la documentación de la API}
		\catch{KeyError}{
		\textbf{regresar: } Error, no se pudo consultar la información}
		\textbf{regresar: } El pronóstico del clima es: X , humedad: x\\ \qquad \qquad \quad Temperatura actual: X$ ^\circ $C, mínima: X$ ^\circ $C, máxima: X$ ^\circ $C Amanecer: X Puesta del sol: X
	\caption{parse\_weather\_info}
	\end{algorithm}

\section{Mejora a futuro}
Podemos hablar de varias mejoras a este proyecto, pues está elaborado justo para ofrecer posteriormente nuevas funcionalidaddes a la clienta, entre ellas podemos enunciar
\begin{itemize}
	\item Aceptar csv con otro tipo de información que no sean coordenadas\\
	Ya que por el momento sólo acepta un formato muy acotado de archivo, donde los parámetros deben estar bien definidos y en cierto orden
	\item Una interfaz gráfica \\
	En especial si se va a emplear directamente en aeropuertos donde las clientas estén consultando la información
	\item Conexión con su sistema\\
	Es casi seguro que si se trata de un \textit{aeropuerto} ya cuenten con un sistema integrado, donde podríamos anexar este nuevo componente
	\item Exportación de datos\\
	Ofrecer que la informació pueda persisitir en caso de que requieran consultar datos pasados o climas históricos
\end{itemize}
\subsection{Cobro por el programa}
Para ponderar el cobro por este pequeño programa hemos decidido seguir la sugerencia de buscar el salario promedio en latinoamérica para unx programadorx, así que, según \textit{talent.com}\footnote{https://mx.talent.com/salary?job=Programador\#} el salario promedio es de 
\[  \$ 92 \text{  por hora} \] 
eso es súper poquito, nos tardamos aproximadamente 10 horas en hacerlo, idear, modelar y plantear la solución fue sencillo, sólo que hubo que revisar sintaxis en python porque no habíamos trabajado en él y eso causó retrasos.
cobraríamos $ \$1,900 $, así cada uno recibiría $ \$ 950 $ por sus horas invertidas.
\end{document}