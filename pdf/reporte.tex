\documentclass[letterpaper]{article}
\usepackage[utf8]{inputenc}
\usepackage[spanish, mexico]{babel}
\usepackage{amssymb, amsmath}
\usepackage{graphicx}
\usepackage[margin=1.5cm,
vmargin={1.5cm,0.7cm},
includefoot]{geometry}
\usepackage{amsthm}
\usepackage{dsfont}
\usepackage{mathtools}
\usepackage{graphicx}
\usepackage{algorithmic}
\usepackage[linesnumbered,ruled,vlined]{algorithm2e}
\begin{document}

\setlength{\unitlength}{1cm}
\thispagestyle{empty}
\begin{picture}(19,3)
\put(-0.5,1.2){\includegraphics[scale=.20]{img/unam1.png}}
\put(16,1){\includegraphics[scale=.29]{img/fciencias1.png}}
\end{picture}

\begin{center}
	\vspace{-114pt}
	\textbf{\large Proyecto 01}\\
	\textbf{ Semestre 2021-1}\\
	Prof. José Galaviz Casas\\
	Ayud. María Ximena Lezama \\
	\textbf{Modelado y programación}\\[0.15cm]
	Kevin Ariel Merino Peña\footnote{Número de cuenta 317031326} Armando Abraham Aquino Chapa\footnote{Número de cuenta n}\\
	\today
\end{center}
\vspace{-10pt}
\rule{19cm}{0.3mm}

\section{Definición del problema}
\section{Análisis del problema}
\section{Selección de la mejor alternativa}

\section{Pseudocódigo}

	\subsection*{CSVReader.py}
	
	\begin{algorithm}
		\SetKwData{Left}{left}\SetKwData{This}{this}
		\SetKwData{Up}{up}
		\SetKwFunction{Union}{Union}
		\SetKwFunction{FindCompress}{FindCompress}
		\SetKwInOut{Input}{Entrada}
		\SetKwInOut{Output}{Salida}
		\SetAlgorithmName{Función}{0}{list of algorithms name}
		\Input{Nombre de un archivo (ruta)}
		\Output{Lista de coordenadas no repetidas}
		\BlankLine
		\While{Archivo(nombre dado) esté abierto}{
		\ForEach{renglones $\leftarrow$ documento}{
				\If{(latitud , longitud ) \textbf{no} están en  la lista}{
				Agreagrlas}
		}
	}
	\caption{read\_no\_repeated\_coordinates}
	\end{algorithm}
	
	\begin{algorithm}
		\SetKwData{Left}{left}\SetKwData{This}{this}
		\SetKwData{Up}{up}
		\SetKwFunction{Union}{Union}
		\SetKwFunction{FindCompress}{FindCompress}
		\SetKwInOut{Input}{Entrada}
		\SetKwInOut{Output}{Salida}
		\SetKwProg{try}{try}{:}{}
		\SetKwProg{catch}{catch}{:}{end}
		\SetAlgorithmName{Función}{0}{list of algorithms name}
		\Input{Nombre de un archivo (ruta)}
		\Output{Lista de diccionarios con vuelos}
		\BlankLine
		\try{}{
				Abrir ruta\\
				\For{linea $ \gets $ archivo}{
				linea $ \gets $ lista}
		}
		\catch{FileNotFoundError}{
			\textbf{muesta}	 Error, escribe una ruta válida\\\textbf{exit}
		}
		\catch{FileExistsError}{
		\textbf{muesta}	 Error, archivo válido\\\textbf{exit}
		}
	\caption{read\_csv\_file}
	\end{algorithm}
	
	\begin{algorithm}
		\SetKwData{Left}{left}\SetKwData{This}{this}
		\SetKwData{Up}{up}
		\SetKwFunction{Union}{Union}
		\SetKwFunction{FindCompress}{FindCompress}
		\SetKwInOut{Input}{Entrada}
		\SetKwInOut{Output}{Salida}
		\SetKwProg{with}{with}{:}{}
		\SetKwProg{catch}{catch}{:}{end}
		\SetAlgorithmName{Función}{0}{list of algorithms name}
		\Input{Nombre de un archivo (ruta)}
		\Output{Lista de cabeceras}
		\BlankLine
		\with{}{
			lector $ \gets $ \textbf{Leer primera linea}( ruta)
		}
	\caption{read\_headers}
	\end{algorithm}
	\newpage
	\subsection*{main.py}
	
	\begin{algorithm}
		\SetKwData{Left}{left}\SetKwData{This}{this}
		\SetKwData{Up}{up}
		\SetKwFunction{longitud}{longitud}
		\SetKwFunction{vf}{validate\_file}
		\SetKwFunction{readHeaders}{read\_headers}
		\SetKwInOut{Input}{Entrada}
		\SetKwInOut{Output}{Salida}
		\SetKwProg{with}{with}{:}{}
		\SetKwProg{catch}{catch}{:}{end}
		\SetAlgorithmName{Función}{0}{list of algorithms name}
		\Input{Nombre de un archivo (ruta) pasados como argumento al programa}
		\BlankLine
		\If{longiutd del argumento no es 2}{
		\textbf{muestra:} Error \\Debe indicar la ruta a un archivo csv\\ \textbf{salir}}
		\If{no coincide la extensión .csv}{\textbf{muestra: } Error, sólo admito archivos csv \\\textbf{salir}}
		cabezera $ \gets $ nombres de listas admitidas\\
		entrada\_cabecera $ \gets $ \readHeaders{argumento[1]}{}\\
		\If{\longitud{entrada\_cabecera} no es igual a \longitud{cabezera}}{
		\textbf{muestra:} ERROR El archivo csv debe tener los siguientes encabezados: origin, destination, origin\_latitude, origin\_longitude, destination\_latitude, destination\_longitude \\
		\textbf{salir}}
		\ForEach{cabeza in entrada\_cabezera}{\If{cabeza no está en cabezera}{\textbf{muestra:} ERROR El archivo csv debe tener los siguientes encabezados: origin, destination, origin\_latitude, origin\_longitude, destination\_latitude, destination\_longitude\\ \textbf{salir} }}
	\caption{validate\_file}
	\end{algorithm}
	
	\begin{algorithm}
		\SetKwData{Left}{left}\SetKwData{This}{this}
		\SetKwData{Up}{up}
		\SetKwFunction{makeapirequestbycoordinates}{make\_api\_request\_by\_coordinates}
		\SetKwFunction{format}{format}
		\SetKwFunction{setdefault}{setdefault}
		\SetKwFunction{vf}{validate\_file}
		\SetKwFunction{readcsvfile}{read\_csv\_file}
		\SetKwFunction{readnorepeatedcoordinates}{read\_no\_repeated\_coordinates}
		\SetKwInOut{Input}{Entrada}
		\SetKwInOut{Output}{Salida}
		\SetKwProg{with}{with}{:}{}
		\SetKwProg{catch}{catch}{:}{end}
		\SetAlgorithmName{Función}{0}{list of algorithms name}
		\Input{Nombre de un archivo (ruta)}
		\BlankLine
		%%%%%%%%%%%%%%%%%%%%%%%%%%%%%%
		\vf{argumentos al correr el programa}{}\;
		entradas $ \gets $ \readcsvfile{argumento al correr el programa}\\
		%%%%%%%5
		solicitudes\_no\_repetidas $ \gets $ \readnorepeatedcoordinates{argumentos al iniciar}\\
		\ForEach{solicitud \textbf{en} solicitudes\_no\_repetidas}{
		peticion $ \gets $ \makeapirequestbycoordinates{solicitud[0], solicitud[1]}\\
		peticiones $ \gets $ \setdefault{solicitud, peticion}\\
		}	
		\ForEach{entrada \textbf{en} entradas}{\textbf{muesta: } Datos del clima ;) con formato bonito }
	\caption{run}
	\end{algorithm}
	
	\newpage
	\subsection*{Weather.py}
	\begin{algorithm}
		\SetKwData{Left}{left}\SetKwData{This}{this}
		\SetKwData{Up}{up}
		\SetKwFunction{makeapirequestbycoordinates}{make\_api\_request\_by\_coordinates}
		\SetKwFunction{get}{get}
		\SetKwInOut{Input}{Entrada}
		\SetKwInOut{Output}{Salida}
		\SetKwProg{with}{with}{:}{}
		\SetKwProg{catch}{catch}{:}{end}
		\SetAlgorithmName{Función}{0}{list of algorithms name}
		\Input{latitud, lontigud}
		\Output{llamada a función parse\_weather\_info }
		\BlankLine
		%%%%%%%%%%%%%%%%%%%%%%%%%%%%%%
		\If{contador $ > $ 59}{contador $ \gets $ 0\\ esperar 1 minuto para continuar}
		\get{url + latitud y longitud dadas}{}\\
		contador $ \gets $ contador + 1
	\caption{make\_api\_request\_by \_coordinates}
	\end{algorithm}
	
	\begin{algorithm}
		\SetKwData{Left}{left}\SetKwData{This}{this}
		\SetKwData{Up}{up}
		\SetKwFunction{getLocalzone}{get\_localzone()}
		\SetKwFunction{fromTimeStamp}{fromtimestamp}
		\SetKwFunction{strftime}{strftime}
		\SetKwInOut{Input}{Entrada}
		\SetKwInOut{Output}{Salida}
		\SetAlgorithmName{Función}{0}{list of algorithms name}
		\Input{Número de fecha y hora (unix)}
		\Output{cadena de texto con hora en formato 12 hrs}
		\BlankLine
		%%%%%%%%%%%%%%%%%%%%%%%%%%%%%%
		\textbf{convierte\_fotante: } numero dado\\
		local\_timezone $ \gets $ \getLocalzone{ }\\
		local\_time $ \gets $ \fromTimeStamp{flotante, local\_timezone}\\
		\textbf{regresar: } local\_time con formato de 12 horas, (CODIGO DEL TIEMPO)
	\caption{formato\_de\_horas}
	\end{algorithm}
	
	\begin{algorithm}
		\SetKwData{Left}{left}\SetKwData{This}{this}
		\SetKwData{Up}{up}
		\SetKwFunction{makeapirequestbycoordinates}{make\_api\_request\_by\_coordinates}
		\SetKwFunction{get}{get}
		\SetKwInOut{Input}{Entrada}
		\SetKwInOut{Output}{Salida}
		\SetKwProg{try}{try}{:}{}
		\SetKwProg{catch}{catch}{:}{end}
		\SetAlgorithmName{Función}{0}{list of algorithms name}
		\Input{respuesta en formato json}
		\Output{llamada a función parse\_weather\_info }
		\BlankLine
		%%%%%%%%%%%%%%%%%%%%%%%%%%%%%%
		\try{ }{
		extraer información del archivo json con las llaves proporcionadas por la documentación de la API}
		\catch{KeyError}{
		\textbf{regresar: } Error, no se pudo consultar la información}
		\textbf{regresar: } El pronóstico del clima es: X , humedad: x\\ \qquad \qquad \quad Temperatura actual: X$ ^\circ $C, mínima: X$ ^\circ $C, máxima: X$ ^\circ $C Amanecer: X Puesta del sol: X
	\caption{parse\_weather\_info}
	\end{algorithm}




\section{Mejora a futuro}
\end{document}